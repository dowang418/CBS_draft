%\documentclass[twocolumn,showpacs,preprintnumbers,superscriptaddress,amsmath,amssymb]{revtex4-1}
\documentclass[twocolumn,showpacs,preprintnumbers,amsmath,amssymb]{revtex4-1}
\usepackage{amssymb}		%math formula
\usepackage{amsmath}
\usepackage{graphicx}
\usepackage[normalem]{ulem}
\usepackage{multirow}
\usepackage{appendix}
\usepackage{CJK}
\usepackage[usenames]{color}
\usepackage{bm}
%\usepackage{hyperref}
\usepackage[colorlinks,linkcolor=blue,urlcolor=blue,citecolor=blue]{hyperref}
% 10.1
\usepackage{booktabs} 	% table!
\usepackage{leftidx}


\makeatletter
\newcommand{\rmnum}[1]{\romannumeral #1}
\newcommand{\Rmnum}[1]{\expandafter\@slowromancap\romannumeral #1@}
\makeatother
\begin{document}
\begin{CJK*} {UTF8} {gbsn}
%\begin{CJK*} {GB} {gbsn}

\title{System size dependence of baryon-strangeness correlations in relativistic heavy-ion collisions from a multiphase transport model}

% my name
\author{Dong-Fang Wang(王东方)}



\author{Song Zhang(张松)}\thanks{Email: song\_zhang@fudan.edu.cn}

\author{Yu-Gang Ma(马余刚)}\thanks{Email:  mayugang@fudan.edu.cn}
\affiliation{Key Laboratory of Nuclear Physics and Ion-beam Application (MOE), Institute of Modern Physics, Fudan University, Shanghai 200433, China}
%\affiliation{Shanghai Institute of Applied Physics, Chinese Academy of Sciences, Shanghai 201800, China}


% Part 0.0: abstrac
\begin{abstract}
The system size dependence of baryon-strangeness  correlations ($C_{BS}$) are studied with a multiphase transport model for various collision systems from 
$\mathrm{^{10}B+^{10}B}$, $\mathrm{^{12}C+^{12}C}$, $\mathrm{^{16}O+^{16}O}$, $\mathrm{^{20}Ne+^{20}Ne}$, $\mathrm{^{40}Ca+^{40}Ca}$, $\mathrm{^{96}Zr+^{96}Zr}$, and $\mathrm{^{197}Au+^{197}Au}$ at RHIC energies $\sqrt{s_{NN}}$ of 200, 39, 27, 19.6, and 7.7 GeV.
%By comparing these results to the latest experimental data from the STAR Collaboration.
The effects of hadronic re-scattering and a combination of different hadrons are playing a leading role in baryon-strangeness correlations. When  the kinetic window is limited to $|y|>3$, these correlations tend to be constant  after final interaction no matter what kind of hadrons subset we chose based on AMPT framework. The correlations smoothly increase with the increasing of baryon chemical potential $\mu_B$ which corresponds to the collision systems (or energy) from the QGP-like phase to the hadron-gas-like phase.
In addition, we investigate the influence of initially nuclear geometry structures through light $\alpha$-clustering reaction systems of $\mathrm{^{12}C+^{12}C}$ and $\mathrm{^{16}O+^{16}O}$ collision, but find the effect is negligible.
These model studies can provide baselines for searching for the signals of QCD phase transition and critical point in heavy-ion collisions.
 	\end{abstract}
\maketitle



	% Part 1.0 introduction
	\section{Introduction}
	% basic intro
	\par
	Quark-gluon plasma (QGP) is proved to create in relativistic heavy-ion collisions~\cite{phase_e1,phase_e2,phase_e3,phase_e4}. 
	This brings a fundamental problem, that is how to identify this hot and dense matter or fully understand the phase diagram of QGP matter.
	Lattice QCD calculations show that the transition from hadronic phase to QGP phase is a crossover at zero baryon-chemical potential $(\mu_{B}=0)$ 
	with a transition temperature $T_{c}\sim166$ MeV~\cite{tc_1,tc_2}.
	%Lots of efforts in the experiment has been made.
	In the years 2010 - 2017, Relativistic Heavy Ion Collider (RHIC) performed the beam energy scan (BES) program~\cite{bes_1,bes_2,bes_3}
to try to answer the above questions, and one of feasible approaches is through studying fluctuations~\cite{koch_re,jeon2003eventbyevent}.
	
	% say something about fluctuations.
	\par
	Theoretical calculations show that fluctuation and correlation of conserved charges are distinctly different in the hadronic and QGP phases~\cite{LuoNST}, 
	and they are experimentally accessible to distinguishing between the two phases~\cite{Adare_2016}.
	Experimental analysis of event-by-event fluctuations of the net-conserved charges like baryon number ($B$), electric charge ($Q$) and strangeness ($S$), in particular, their higher-order cumulants  have been announced at RHIC~\cite{exp_1,exp_2} and LHC~\cite{exp_3,exp_4}.
	
	
	One of the event-by-event fluctuation observables was proposed by Koch~\cite{Koch_origin}, namely the baryon-strangeness correlation coefficient, 
	\begin{eqnarray}%{equation}
	\begin{aligned}
	 C_{BS} = -3\frac{\left\langle BS \right\rangle - \left\langle B \right\rangle \left\langle S \right\rangle}{\left\langle S^{2}  \right\rangle - \left\langle S \right\rangle^{2}},
	 \end{aligned}
	\end{eqnarray}%{equation}
	where $B$ and $S$ are the net-baryon number and net-strangeness in one event, respectively.
	The average values over the whole event ensemble are denoted by $\left\langle \cdot \right\rangle$.
	This correlation is proposed as a tool to specify the nature (ideal QGP or strongly coupled QGP or hadronic matter) of the highly compressed and heated matter created in heavy ions collisions.
	After that, people used some specific models, such as (2 + 1) Polyakov Quark Meson Model~\cite{u_model1}, 
	hadron resonance gas model~\cite{u_model2,u_model3}, UrQMD~\cite{u_model4,Haussler_2007,Luo_UrQMD} as well as AMPT model~\cite{Jin_2008}
	 to study the fluctuations and compare them with Lattice QCD results~\cite{lattice_1,lattice_2}.
	
	\par
	%Not only the fluctuation of conserved charges has attracted people's attention, but small systems~\cite{small_sys_LHC,small_sys_1} have also been widely discussed in recent years.
On the other hand, the investigation on a small system has underwent for some years in experiments and theoretical works~\cite{small_sys_LHC,small_sys_1}, and some proposals of system scan were promoted to study the possible signal  of QGP matter in the small system as well as the initial state effects on the final state~\cite{SHuang2020sysScan,PRC2029sysScanLHC,ZHANG2020135366}.
	For example, the ALICE collaboration reported the enhanced production of multi-strange hadrons in high-multiplicity proton-proton collisions~\cite{smallSystemALICE2017}. In the same context, we think that the baryon-strangeness correlation which is related to the QGP phase transition may also be sensitive to the fluctuations from a small system to large systems in heavy-ion collisions.
	
	%With more and more collision system proposed in the experiments after BES program, naturally, we want to understand the system dependence of physical quantities such as collective flow.
	%From the existing experimental results, we found, in small system, the evidence of QGP matter could also been measured which is contrary to expectations. 
	

	\begin{table*}[]
\scriptsize
\centering
\caption{AMPT input parameters and $\left\langle \mathrm{N_{part}}\right\rangle$ values of different collision systems.}
\label{AMPT_info}

	\begin{tabular}{cc|cc|cc|cc}
\toprule
\multicolumn{2}{c}{} & \multicolumn{2}{c}{$\sqrt{s_{NN}}$ = 200GeV} & \multicolumn{2}{c}{$\sqrt{s_{NN}}$ = 19.6GeV} & \multicolumn{2}{c}{$\sqrt{s_{NN}}$ = 7.7GeV} \\
\cmidrule(r){3-4} \cmidrule(r){5-6} \cmidrule(r){7-8}
System & $\mathrm{\it{b_{max}}[\it{fm}]}$
&$\left\langle \mathrm{N_{part}}\right\rangle$ &Event counts
&$\left\langle \mathrm{N_{part}}\right\rangle$ &Event counts
&$\left\langle \mathrm{N_{part}}\right\rangle$ &Event counts      \\
\hline
$\mathrm{\leftidx{^{10}}B} + \mathrm{\leftidx{^{10}}B}$		&1.15619		&14.8  &7$\times 10^{4}$    	&13.2  &12$\times 10^{4}$  &13.1  &16$\times 10^{4}$\\
$\mathrm{\leftidx{^{12}}C} + \mathrm{\leftidx{^{12}}C}$		&1.22864		&18.7  &10$\times 10^{4}$    	&16.8  &6$\times 10^{4}$ 	&16.7  &10$\times 10^{4}$\\
$\mathrm{\leftidx{^{16}}O}+\mathrm{\leftidx{^{16}}O}$		&1.35229		&25.5   &10$\times 10^{4}$	&23.1  &4$\times 10^{4}$	&23.0  &10$\times 10^{4}$ \\
$\mathrm{\leftidx{^{20}}Ne}+\mathrm{\leftidx{^{20}}Ne}$		&1.45671		&32.8  &2$\times 10^{4}$	&30.0  &4$\times 10^{4}$	&29.8  &2$\times 10^{4}$\\
$\mathrm{\leftidx{^{40}}Ca}+\mathrm{\leftidx{^{40}}Ca}$		&1.83534		&69.3  &2$\times 10^{4}$	&65.0  &1$\times 10^{4}$	&64.9  &1$\times 10^{4}$ \\
$\mathrm{\leftidx{^{96}}Zr}+\mathrm{\leftidx{^{96}}Zr}$ 		&2.45727		&174.2 &2$\times 10^{4}$	&167.3  &2$\times 10^{4}$	&166.9 &3$\times 10^{4}$\\
$\mathrm{\leftidx{^{197}}Au}+\mathrm{\leftidx{^{197}}Au}$		&3.1226		&364.1 &1$\times 10^{4}$	&354  &3$\times 10^{4}$	&353.8  &3$\times 10^{4}$\\
\bottomrule
\end{tabular}
\end{table*}
	
		
	
	% this work CBS + system scan + energy scan.
	\par
	In this work, we adopt a multiphase transport model to study colliding system size influence on baryon-strangeness correlations $C_{BS}$.
	By tuning the colliding energies of two nuclei, we investigate the energy dependence of correlations $C_{BS}$.
		In addition, we also study the maximum rapidity acceptance $y_{\text{max}}$ dependence as well as the influence of initially nuclear geometry structure.
	
	% this paper structure
	The paper is organized as follows. 
	First, in Sec.\ref{sec:model}, we give a short introduction to a multiphase transport model and some input parameters used in this study are presented
	and briefly manifest the physical picture of baryon-strangeness correlations.
	Next, the baryon-strangeness correlations as a function of system size, energy, and $y_{\text{max}}$ are discussed in Sec.\ref{sec:result}. 
	Finally, a brief summary is presented in Sec.\ref{sec:summary}.
	
	
	\section{Model and methodology}
	\label{sec:model}
	\subsection{Brief introduction to AMPT model}


		A multi-phase transport model (AMPT), which is a hybrid dynamic model, is employed to calculate different collision systems.
	The AMPT model can describe the $p_{T}$ distribution of charged particles \cite{xujun,suppressionhighpt,Ye_2017,JinXH} and their elliptic flow of Pb+Pb collisions
	at $\sqrt{s_{NN}} = 2.76$ TeV, as measured through the LHC-ALICE Collaboration. The model includes four main components to describe
	the relativistic heavy-ion collision process: the initial condition which is simulated using the Heavy Ion Jet Interaction Generator (HIJING) model~\cite{HIJING-1,HIJING-2}, the partonic interaction which is described by Zhang's Parton Cascade (ZPC) model~\cite{ZPCModel}, the hadronization process which is went through by a Lund string fragmentation or coalescence model, and the hadronic re-scattering process which is treated by  A Relativistic Transport (ART) model~\cite{ARTModel}. There are two versions of AMPT: 1) the AMPT version with a string melting mechanism, in which a partonic phase is generated from excited strings in the HIJING model, where a simple quark coalescence model is used to combine the partons into hadrons; and 2) the default AMPT version which only undergoes a pure hadron gas phase. 
	The AMPT model succeeds to describe extensive physics topics for relativistic heavy-ion collisions at the RHIC~\cite{AMPT_origin} as well as the LHC~\cite{AMPTGLM2016} energies, 
e.g. for hadron HBT correlations~\cite{AMPTHBT}, di-hadron azimuthal correlations~\cite{AMPTDiH,WangHai}, collective flows~\cite{STARFlowAMPT,AMPTFlowLHC}, strangeness productions~\cite{JinXH,SciChinaJinS} as well as chiral magnetic effects and so on \cite{Zhao,Huang,Wang,XuZW}.
The details of AMPT can be found in Ref.~\cite{AMPT_origin}.

		In the AMPT model, impact parameter $b$, which is the transverse distance between the centers of the two collided nuclei, can determine the collision centrality. In addition, the number of participants is also related to the centrality or impact parameter. In this study, we adopt the AMPT parameters as  suggested in Ref.~\cite{Ye_2017}.  The calculated  collision systems and energies, their corresponding maximum impact parameters, the number of participants $N_{Part}$ and the number of events  are listed in Table~\ref{AMPT_info}.
		
		
		‘
	\subsection{Baryon-strangeness correlations $C_{BS}$}
	
	How to diagnose or find a very clear probe to distinguish QGP matter is always  the core issue of understanding the QGP phase transition in relativistic heavy-ion collisions.
	This correlation coefficient $C_{BS}$ has its own advantage because it is on  conserved quantities, which is less affected due to  uncertainty from hadronization.
	
	{\color{blue}{\sout {In terms of physical picture, on the one hand,}}} Under ideal QGP assumption, where the basic degrees of freedom are weakly interacting quarks and gluons at high temperature,  $\left\langle S \right\rangle$ keeps $0$ and $C_{BS}$ can be written as $C_{BS} = -3 \frac{\left\langle BS \right\rangle}{\left\langle S^{2}\right\rangle}=1$,
	noting that the baryon number of a quark is $1/3$ and the strangeness of a strange quark is $-1$ and the strangeness only carried by $s$ quark~\cite{Koch_origin}.
	However, this feature is different from a hadron gas, and this coefficient strongly depends on the hadronic environment.
	Under an assumption of uncorrelated multiplicities, $C_{BS}$ can be written as~\cite{Koch_origin}
\begin{equation}
C_{B S} \approx 3 \frac{\langle\Lambda\rangle+\langle\bar{\Lambda}\rangle+\cdots+3\left\langle\Omega^{-}\right\rangle+3\left\langle\bar{\Omega}^{+}\right\rangle}{\left\langle K^{0}\right\rangle+\left\langle\bar{K}^{0}\right\rangle+\cdots+9\left\langle\Omega^{-}\right\rangle+9\left\langle\bar{\Omega}^{+}\right\rangle}.
\end{equation}
	In actual calculation~\cite{Koch_origin},
\begin{equation}
C_{B S}=-3 \frac{\sum_{n} B^{(n)} S^{(n)}-\frac{1}{N}\left(\sum_{n} B^{(n)}\right)\left(\sum_{n} S^{(n)}\right)}{\sum_{n}\left(S^{(n)}\right)^{2}-\frac{1}{N}\left(\sum_{n} S^{(n)}\right)^{2}},
\end{equation}
where $B$ and $S$ denote the net baryon number and net strangeness observed for a given event, respectively, and 
$N$ is the total number of events. In the study, some attentions should also be payed for 
the statistical error as suggested in reference~\cite{Luo_UrQMD,Luo_2012}(see the Appendix).	
%\begin{equation}
%
%\end{equation}
	\section{Results and discussion}
	\label{sec:result}
		
		The combination of hadrons would play an important role in the quality of the baryon-strangeness correlations $C_{BS}$. To investigate this effect, the net-baryon $B$ versus net-strangeness $S$ distribution was calculated and presented in Fig~\ref{Fig1_BS_plane}.
		We chose two combinations of hadrons for baryon-strangeness correlation calculation in this figure:
		(\Rmnum{1}), $p$+$n$+$\Lambda$+$\Sigma^{\pm}$+$\Xi^{\pm}$+$\Omega^{-}$+$K$, (\Rmnum{2}) $p$+$n$+$K$, 
	where both the particles and anti-particles are included
	with kinetic windows $0.1<p_{T}<3.0$ GeV/c and $|y|<0.2$.
	In this figure, we {\color{blue}{\sout {intuitively}}} observe that the baryon-strangeness distribution is more concentrated on the center in the case of without the effect of hadronic re-scattering where $\left\langle B \right\rangle$ and $\left\langle S \right\rangle$ are approximately equal to 0, which leads to stronger correlations.
	
	When we {\color{blue}{\sout {artificially}}} count more strange baryons (and anti-baryons), the distribution would be stretched into elliptical distribution and lead to finite negative correlations, usually 
	represent as:
	\begin{equation}
\rho_{B, S}=\frac{\operatorname{cov}(B, S)}{\sigma_{B} \sigma_{S}}=\frac{\langle(B-\langle B\rangle)(S-\langle S\rangle)\rangle}{\sqrt{\left\langle B^{2}\right\rangle-\langle B\rangle^{2}} \sqrt{\left\langle S^{2}\right\rangle-\langle S\rangle^{2}}}.
\end{equation}
The more strange baryons were used, the more negative correlation was presented between net-baryon $B$ and net-strangeness $S$.
	
		\begin{figure}[htb]
				\includegraphics[angle=0,scale=0.45]{./Fig1_CBS_plane.eps}
				\caption{The correlation between net-baryon $B$ and net-strangeness $S$ for two different subsets of hadrons in the most central (0$-$5\%) $\mathrm{^{197}Au+^{197}Au}$ collisions at $\sqrt{s_{NN}} = 200$ GeV with the string melting AMPT framework.}
				\label{Fig1_BS_plane}
	\end{figure}
	%\subsection{System and energy scan results}

	In this paper, we would focus on (\Rmnum{1}) and (\Rmnum{2}) hadrons combination for calculating correlations. In the following, we investigate the system size dependence of $C_{BS}$ under the effect of hadronic re-scattering. Figure~\ref{Fig2_Sys_E_scan_CBS} shows the system size dependence for all particles in 0-5\% $\mathrm{^{10}B+^{10}B}$, $\mathrm{^{12}C+^{12}C}$, $\mathrm{^{16}O+^{16}O}$, $\mathrm{^{20}Ne+^{20}Ne}$, $\mathrm{^{40}Ca+^{40}Ca}$, $\mathrm{^{96}Zr+^{96}Zr}$, and $\mathrm{^{197}Au+^{197}Au}$ collisions at $\sqrt{s_{NN}}$= 200, 19.6, and 7.7 GeV from the AMPT model. 
	
	In the case without hadron re-scattering where the hadronized system just undergoes a partonic phase, the baryon-strangeness correlations $C_{BS}$ keep constant at $\sqrt{s_{NN}}$= 200 and 19.6 GeV as collision system size increases. As collision energy increases, $C_{BS}$ approaches  the value in an ideal QGP assumption ($C_{BS}$=1). At $\sqrt{s_{NN}}$= 7.7 GeV, $C_{BS}$ almost keeps as constant but shows a slightly decreasing trend with system size.
	
	For the case with hadron re-scattering at $\sqrt{s_{NN}}$= 200 and 19.6 GeV, the baryon-strangeness correlations $C_{BS}$ show similar behavior. {\color{blue}{\sout {Similarly,}}} At $\sqrt{s_{NN}}$= 7.7 GeV, the pattern of $C_{BS}$ changing with the system size is not completely flat. {\color{blue}{\sout {It's obvious that}} }The re-scattering process would erase the signal of partonic matter which is consistent with the earlier AMPT study~\cite{Jin_2008}.
{\color{blue}{\sout {Considering that}} This dependence is also related to rapidity distribution, thus} we also present baryon and strangeness $dN/dy$ in $\mathrm{^{197}Au+^{197}Au}$ collisions at RHIC energies $\sqrt{s_{NN}}$ = 200, 19.6, and 7.7 GeV based on the string melting AMPT model, as shown in figure~\ref{Fig3_dNdy}.
	 
 We find that the net-baryon $B$ rapidity distribution becomes more concentrated in the middle rapidity when collision energy decreases as shown in Fig.~\ref{Fig3_dNdy}(a), (b) and (c).
 At $\sqrt{s_{NN}}$ = 19.6 and 7.7 GeV, the low collision energy leads to positive baryon $B^{+}$ is much larger than negative baryon $B^{-}$ (almost anti-proton).
After the hadron re-scattering process, because of decay particles contribution, net-baryon $B$ shows a little higher than the value without hadron re-scattering{\color{blue}{\sout {, represented as a hollow marker in this figure}}}. At $\sqrt{s_{NN}}$ = 200GeV, non-Gaussian $B$ rapidity distribution has also been found in Ref.~\cite{Lin_2017xkd}.

 Figure~\ref{Fig3_dNdy}(e) and (f) display net-strangeness $S$ rapidity distribution at 19.6 and 7.7 GeV, respectively,  where $S$ is always negative.
 At $\sqrt{s_{NN}}$ = 200 GeV,  $S$ turns to positive at mid rapidity as shown in Fig~\ref{Fig3_dNdy}(d).
{\color{blue}{\sout {From this,}} As energy increases, net-baryon $B$ decreases at the chosen rapidity region, and the baryon-strangeness correlations $C_{BS}$ is closer to 1 manifesting the system close to the QGP state}. 
The more net-baryon $B$ the collision system has, the state would be more close to the hadron gas phase {\color{blue}with larger $C_{BS}$. {\sout {and $C_{BS}$ would also be larger}}}.

As energy decreases, net-baryon $B$ and net-strangeness $S$ are growing a more sharp peak as plotted in Fig.~\ref{Fig3_dNdy}.
{\color{blue}{\sout {Predictably,}} If we only slightly changes} the kinetic window size, the final result of $C_{BS}$ would change a lot.
Thus, we might draw a conclusion that $C_{BS}$ {\color{blue}{\sout {may}}} also be affected by kinetic windows because of this non-flat rapidity distribution.

		\begin{figure*}[htb]
				\includegraphics[angle=0,scale=0.85]{./Fig2_one_sysScan_4.eps}
				\caption{The baryon-strangeness correlations $C_{BS}$ versus $\left\langle \mathrm{N_{part}}\right\rangle$ at  $\sqrt{s_{NN}}$ =200, 20 and 7.7 GeV in the most central (0$-$5\%) $\mathrm{^{10}B+^{10}B}$, $\mathrm{^{12}C+^{12}C}$, $\mathrm{^{16}O+^{16}O}$, $\mathrm{^{20}Ne+^{20}Ne}$, $\mathrm{^{40}Ca+^{40}Ca}$, $\mathrm{^{96}Zr+^{96}Zr}$, and $\mathrm{^{197}Au+^{197}Au}$ collisions at RHIC energies $\sqrt{s_{NN}}$ = 200 (a), 19.6 (b), and 7.7 GeV (c) in the string melting AMPT framework.				}
				\label{Fig2_Sys_E_scan_CBS}
	\end{figure*}

		\begin{figure*}[htb]
				\includegraphics[angle=0,scale=0.85]{./Fig3_six_pad_6.eps}
				\caption{The 
				AMPT results of positive baryon (strangeness) $B^{+}$ ($S^{+}$), negative baryon (strangeness) $B^{-}$ ($S^{-}$) and net-baryon (-strangeness) $B$ ($S$) $dN/dy$ for identified $p$, $n$, $\Lambda$, $\Sigma^{\pm}$, $\Xi^{\pm}$, $\Omega^{-}$, and  $K$ in
				$\mathrm{^{197}Au+^{197}Au}$ collisions at RHIC energies $\sqrt{s_{NN}}$ = 200 (a,d), 19.6 (b,e), and 7.7 (c,f) GeV based on the string melting AMPT framework. Here the kinematics window is $|y|<0.2$.
				}
				\label{Fig3_dNdy}
	\end{figure*}
	
	


\par
% y_max results

		
	\begin{figure}[htb]
				\includegraphics[angle=0,scale=0.43]{./Fig4_together_0.eps}
				\caption{
				The maximum rapidity acceptance ($|y|<y_{\text{max}}$) dependence of the correlation coefficient $C_{BS}$
				in $\mathrm{^{197}Au+^{197}Au}$ collisions at $\sqrt{s_{NN}} = 200$ GeV with the string melting AMPT framework.
				Two different subsets of hadrons are adopted to show different dependencies.
				The two dashed lines indicate theoretical estimate of simple QGP ($C_{BS}$=1) and hadron gas ($C_{BS}$=0.66) at chemical freeze-out condition of $T$ =170 MeV and $\mu_{b}$ = 0, respectively.  }
				\label{Fig4_ymax}
	\end{figure}
	
				\begin{figure}[htb]
				\includegraphics[angle=0,scale=0.43]{./Fig5_two_pad_Num_Den.eps}
				\caption{
				The maximum rapidity acceptance ($|y|<y_{\text{max}}$) dependence of the numerator ($C_{11}^{BS}$) (a) and the denominator ($C_{2}^{S}$) (b) of $C_{BS}$
				in $\mathrm{^{197}Au+^{197}Au}$ collisions at $\sqrt{s_{NN}} = 200$ GeV with the string melting AMPT framework.
				Two different subsets of hadrons are adopted to show different dependencies.
				}
				\label{Fig5_Num_Den}
	\end{figure}
	
		\begin{figure}[htb]
				\includegraphics[angle=0,scale=0.43]{./Fig6_two_pad_E_sys_scan.eps}
				\caption{
				{\color{blue}(a)} The correlation coefficient $C_{BS}$ in the most central (0$-$5\%) $\mathrm{^{197}Au+^{197}Au}$ collisions is shown as a function of $\sqrt{s_{NN}}$; 
				%We adopt $\sqrt{s_{NN}}$=11.5, 19.6, 27, 39, and 200 GeV.
				{\color{blue}(b)} The correlation coefficient $C_{BS}$ for a hadron gas at freeze-ou is shown as a function of the baryon chemical potential $\mu_{B}$ in the most central (0$-$5\%) collision at different collision systems and energy.
				For $C_{BS}$ system scan, we chose
				 $\mathrm{^{10}B+^{10}B}$, $\mathrm{^{12}C+^{12}C}$, $\mathrm{^{16}O+^{16}O}$, $\mathrm{^{20}Ne+^{20}Ne}$, $\mathrm{^{40}Ca+^{40}Ca}$, $\mathrm{^{96}Zr+^{96}Zr}$, and $\mathrm{^{197}Au+^{197}Au}$ collisions at $\sqrt{s_{NN}}$ = 200GeV.
				 For $C_{BS}$ energy scan, the choice of energy is $\sqrt{s_{NN}}$=11.5, 19.6, 27, 39, and 200 GeV.
				}
				\label{Fig6_E_sys_mub}
	\end{figure}
		
		We also present the result for Au+Au collisions at $\sqrt{s_{NN}}$ = 200 GeV as a function of the rapidity acceptance range $y_{\text{max}}$ from the AMPT model.
	From Fig.~\ref{Fig4_ymax}, based on two different combinations of hadrons, we observe two different $y_{\text{max}}$ dependence.
	If we chose $p, n, K$, and their antiparticles, the coefficient tends to increase with  $y_{\text{max}}$.
	However, when we chose all hadrons we list before, the coefficient tends to decrease.
	Both two cases $C_{BS}$ tends  asymptotically to be a constant after $y_{\text{max}}>3$.
	Also, the choice of hadron combination has no effect on the result at large $y_{\text{max}}$ if there is the hadronic re-scattering stage.
	It could be regarded as the consequence that baryon number and strangeness are conserved quantities. In previous study~\cite{Koch_origin}, the correlation coefficient $C_{BS}$ will first increase with $y_{\text{max}}$  
	and reach a maximum value at a certain $y_{\text{max}}$ before it drops to 0.
	

	%  9.19
	{\color{blue}To understand this phenomenon}, in Fig.~\ref{Fig5_Num_Den}(a) and (b), we show the maximum rapidity acceptance ($|y|<y_{\text{max}}$) dependence of the numerator ($C_{11}^{BS}= \left\langle BS \right\rangle - \left\langle B \right\rangle \left\langle S \right\rangle$) and the denominator ($C_{2}^{S}= \left\langle  S^{2} \right\rangle -\left\langle S \right\rangle^{2}$) of $C_{BS}$ in $\mathrm{^{197}Au+^{197}Au}$ collisions at $\sqrt{s_{NN}} = 200$ GeV with the string melting AMPT model, respectively. In the case of \Rmnum{2} ($p, n, K$), both $-3*C_{11}^{BS}$ and the $C_{2}^{S}$, they gradually tend to a constant value as $y_{\text{max}}$ increases as shown in Fig~\ref{Fig5_Num_Den}(a) and (b), respectively. However, in the case of \Rmnum{1}, $-3*C_{11}^{BS}$ increases with $y_{\text{max}}$ and then decreases to a constant value with hadron re-scattering process. {\color{blue}{\sout {From this phenomenon we can see,}} }Thus, the value of $C_{11}^{BS}$ is the dominant factor affecting the correlation coefficient $C_{BS}$.
	
		
	\par
	%11.5, 19.6, 27,39,200GeV
	The energy scan results of $C_{BS}$ by using the AMPT model are shown in Fig.~\ref{Fig6_E_sys_mub}(a) at $\sqrt{s_{NN}}$ = 11.5, 19.6, 27, 39, and 200 GeV in $\mathrm{^{197}Au+^{197}Au}$ central collisions.
	$C_{BS}$ shows strong dependence on collision energy.
	As energy increases, $C_{BS}$ goes down to 0.6 at the top RHIC energy.
	This result is also below the expected value for an ideal QGP phase which has also been mentioned~\cite{Haussler_2007}.
	

	\par
	Figure~\ref{Fig6_E_sys_mub}(b) show $C_{BS}$ as a function of baryon chemical potential $\mu_{B}$ at chemical freeze-out at $\sqrt{s_{NN}}$= 200 GeV.
		The extracted $\mu_{B}$ based on the thermal model is shown in our previos paper~\cite{wdf}.
	At given collision energy, $\mu_{b}$ increases with system size and we could find a similar trend as shown Fig.~\ref{Fig2_Sys_E_scan_CBS}(a).
	The correlation coefficient $C_{BS}$ with the hadronic re-scattering process slightly increases with $\mu_{b}$, which is consistent with the previous conclusion~\cite{Koch_origin}.
	Meanwhile, in Fig.~\ref{Fig6_E_sys_mub}(b), the collision energy dependence of $C_{BS}$ was also displayed.
	At a given system, as energy decreases, there will be more net baryons in the collision system leading to correlation $C_{BS}$ enhancement. 
	{\color{blue}{\sout {It is interesting that }}The} correlation coefficients present a smooth baryon chemical potential dependence if the collision systems and collision energies are characterized by the potential.
	
	\par
	Finally, we investigate the effect of $\alpha$ cluster distribution on the correlation coefficient $C_{BS}$.
	%In principle, besides thermal fluctuations, there are many other sources and types of fluctuations, such as quantum fluctuations.
	In previous studies, some works have proposed that the signatures of $\alpha$-clustering structure in light nuclei could be observed from heavy-ion collisions even at ultra-relativistic energies~\cite{a_cluster_origin,AlphaModelHe1,AlphaClusterHIC-1,PhysRevC95064904,Zhang2,ChengYL,PhysRevC99044904,PhysRevC101021901,HeJJ,MaLong}. In this context, 
	{\color{blue}{\sout { we would like to check}}}we check the influence of the fluctuation of the initially nuclear geometry structure on the correlation coefficient $C_{BS}$. 
	{\color{blue}{\sout {As we know,}} In this study,} $^{12}\mathrm{C}$ might be considered as a three-$\alpha$ clustered triangle structure and $^{16}\mathrm{O}$ with four-$\alpha$ clustered tetrahedron structure.
	
	In figure.~\ref{Fig7}, we chose three different energies $\sqrt{s_{NN}}$ = 10, 200, and 6370 GeV which are represented by different $\left\langle \mathrm{N_{part}}\right\rangle$
	to illustrate $\alpha$ cluster effects for $\mathrm{^{12}C+^{12}C}$ as well as $\mathrm{^{16}O + ^{16}O}$ collisions. However, 
	nearly no difference is visible between the Woods-Saxon and the $\alpha$-clustering structures on the correlation results. The results illustrate that baryon-strangeness is insensitive to initial nucleon distribution, which can in turn help to isolate other ingredients for affecting baryon-strangeness correlation, such as the hadronic re-scattering effect as discussed in this work.  
	
	
		\begin{figure}[htb]
		\includegraphics[angle=0,scale=0.42]{Fig7.eps}
					\caption{The correlation coefficient $C_{BS}$ as a function of energy (represented by the number of participants $\langle \mathrm{N_{part}}\rangle$) 
				in the most central (impact parameter $b=0$) collisions of $\mathrm{^{12}C+^{12}C}$ and $\mathrm{^{16}O + ^{16}O}$ systems at $\sqrt{s_{NN}}$ = 10, 200, and 6370 GeV.}
				\label{Fig7}
	\end{figure}

	
\section{summary}
\label{sec:summary}	


In this work, we study the system and energy dependence of the baryon-strangeness correlation coefficient in the framework of AMPT model. 
The hadronic re-scattering process partly dissipated the baryon-strangeness correlations as expected and weak decay contributions for strangeness or the count of baryons may have an effect on final results and need to be further investigated.
The combination of different hadrons also affect the results significantly. We also find when the maximum rapidity acceptance $y_{\text{max}}>3$, these coefficients are independent on the combination of different hadrons in the final state based on AMPT model. It is found that the correlation coefficients can be grouped if the collision systems and collision energies are characterized by the baryon chemical potential.
 In addition, we checked the effect of initial nucleon distribution, specifically for $\alpha$-clustering structure in $^{12}$C and $^{16}$O nuclei, to the related results but find negligible effect on the baryon-strangeness correlation.
 %seems not sensitive to this fluctuation.


		\begin{acknowledgements}
	
This work was supported in part by the National Natural Science Foundation of China under contract Nos. 11890714, 11875066, 11421505, and 11775288, and the National Key R\&D Program of China under Grant Nos. 2016YFE0100900 and 2018YFE0104600.

	\end{acknowledgements}
	
	
	
	\section{Appendix}
		  \begin{appendices} 
	\section{observable} 
	The joint cumulant of several random variables $X_{1}, ..., X_{n}$ is defined by a similar cumulant generating function
	\begin{equation}
K\left(t_{1}, t_{2}, \ldots, t_{n}\right)=\log E\left(\mathrm{e}^{\sum_{j=1}^{n} t_{j} X_{j}}\right).
\end{equation}
	 A consequence is that
	 \begin{equation}
\kappa\left(X_{1}, \ldots, X_{n}\right)=\sum_{\pi}(|\pi|-1) !(-1)^{|\pi|-1} \prod_{B \in \pi} E\left(\prod_{i \in B} X_{i}\right)
\end{equation}
where $\pi$ runs through the list of all partitions of $\{ 1, ..., n \}$, B runs through the list of all blocks of the partition $\pi$, and $|\pi|$ is the number of parts in the partition.
In this analysis, we use $B$ and $S$ to represent the net-baryon number and net-strangeness in one event, respectively.
The deviation of $B$ and $S$ from their mean value are expressed by $\delta B = B - \left\langle B \right\rangle$ and $\delta S = S -\left\langle S \right\rangle$, respectively. 
As mentioned above, we use $\left\langle \cdot \right\rangle$ represent expected value.
 According eq.2
\begin{equation}
\begin{aligned}
%C(\delta B,\delta S) &=  \left\langle \delta B \delta S \right\rangle\\
%			&= \left\langle BS \right\rangle - \left\langle B \right\rangle \left\langle S \right\rangle\\
%C(\delta S,\delta S) &=  \left\langle \delta S \delta S \right\rangle\\
%			&= \left\langle S^{2} \right\rangle - \left\langle S \right\rangle^{2}\\
C(\delta B,\delta S) &=  \left\langle \delta B \delta S \right\rangle = \left\langle BS \right\rangle - \left\langle B \right\rangle \left\langle S \right\rangle\\
C(\delta S,\delta S) &=  \left\langle \delta S \delta S \right\rangle = \left\langle S^{2} \right\rangle - \left\langle S \right\rangle^{2}\\
			\end{aligned}
\end{equation}
Thus, the baryon-strangeness correlation coefficient:
\begin{equation}
\begin{aligned}
C_{BS} = -3\frac{C(\delta B,\delta S)}{C(\delta S,\delta S)} = -3\frac{\left\langle BS \right\rangle - \left\langle B \right\rangle \left\langle S \right\rangle}{\left\langle S^{2}  \right\rangle - \left\langle S \right\rangle^{2}}
			\end{aligned}
\end{equation}
\end{appendices}
	
	
	  \begin{appendices} 
      \section{ error  } 
	 In paper~\cite{Luo_UrQMD} appendix, the author detailed display how to calculate statistical error in the way of the covariance of the multivariate moments.
	 In Eq(A2), 
	 \begin{equation}
\operatorname{cov}\left(f_{i, j}, f_{k, h}\right)=\frac{1}{N}\left(f_{i+k, j+h}-f_{i, j} f_{k, h}\right)
\end{equation}
higher-order terms must be introduced for calculating the covariance.
	 So we give all the items for those are necessary for calculating the error thought the table below. 
	 Form the equation we know the error is proportional to $1/\sqrt{N}$, however, the corresponding event statistics we use are relatively small, 
	 the statistical errors of results would be large.
	 	\begin{table*}[]
\scriptsize
\centering
\caption{This table list all the variables needed to calculate the results and statistical errors in Fig.~\ref{Fig2_Sys_E_scan_CBS}(a) at $\sqrt{s_{NN}} = 200$ GeV with hadronic re-scattering process in AMPT framework.}
\label{error_info}

	\begin{tabular}{|c|c|c|c|c|c|c|c|c|c|c|c|c|c|}
\toprule
%\multicolumn{2}{c}{} & \multicolumn{2}{c}{$\sqrt{s_{NN}}$ = 200GeV} & \multicolumn{2}{c}{$\sqrt{s_{NN}}$ = 20GeV} & \multicolumn{2}{c}{$\sqrt{s_{NN}}$ = 7.7GeV} \\
%\cmidrule(r){3-4} \cmidrule(r){5-6} \cmidrule(r){7-8}
System & Events
&$\left\langle B \right\rangle$ & $\left\langle S \right\rangle$
&$\left\langle BS \right\rangle$  & $\left\langle S^{2} \right\rangle$ 
&$\left\langle S^{3} \right\rangle$  &$\left\langle S^{4} \right\rangle$
&$\left\langle BS^{2} \right\rangle$  &$\left\langle BS^{3} \right\rangle$
&$\left\langle B^{2} \right\rangle$  &$\left\langle B^{2}S \right\rangle$
&$\left\langle B^{2}S^{2} \right\rangle$  & error
      \\
\hline
$\mathrm{\leftidx{^{10}}B} + \mathrm{\leftidx{^{10}}B}$		&69800&0.0736533&0.0896132&-0.359728&1.49643&0.513395&9.36554&0.00531519&-2.32606&1.09706&0.0775501&2.47291&0.0233956 \\
$\mathrm{\leftidx{^{12}}C} + \mathrm{\leftidx{^{12}}C}$		&99800&0.0812625&0.109599&-0.493968&1.95445&0.898196&14.9372&-0.0448697&-3.89148&1.45303&0.119419&4.09048&0.0178936\\
$\mathrm{\leftidx{^{16}}O}+\mathrm{\leftidx{^{16}}O}$		&100000&0.12687&0.16194&-0.66834&2.72044&1.69218&27.0776&-0.00802&-6.71508&2.02597&0.2679&7.38878&0.0153406 \\
$\mathrm{\leftidx{^{20}}Ne}+\mathrm{\leftidx{^{20}}Ne}$		&20000&0.21585&0.20645&-0.90485&3.62225&2.55395&45.4021&0.18435&-11.515&2.81145&0.28175&13.1649&0.0283403\\
$\mathrm{\leftidx{^{40}}Ca}+\mathrm{\leftidx{^{40}}Ca}$		&20000&0.57985&0.49615&-1.86935&8.62085&12.5274&234.946&2.55565&-50.8957&6.67815&1.27765&67.8139&0.00965114 \\
$\mathrm{\leftidx{^{96}}Zr}+\mathrm{\leftidx{^{96}}Zr}$ 		&19900&2.08171&1.28734&-3.71834&24.8124&93.0328&1867.91&34.6337&-295.664&22.621&3.92558&588.743&0.0124063\\
$\mathrm{\leftidx{^{197}}Au}+\mathrm{\leftidx{^{197}}Au}$		&10000&5.8683&2.5061&-0.6767&58.6817&434.255&10679.6&258.236&-324.446&77.1819&14.6115&4089.6&0.496962\\
\bottomrule
\end{tabular}
\end{table*}

\end{appendices} 
	\end{CJK*}	
		%\bibliographystyle{unsrt}
		%\bibliographystyle{abbrv}
		\bibliography{no}
		
	
	
\end{document}

